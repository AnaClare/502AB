\documentclass{article}
\usepackage[utf8]{inputenc}
\usepackage{amsfonts,amsmath,mathtools}

\title{Math 502AB - Lecture 19}
\author{Dr. Jamshidian}
\date{October 30, 2017}
\begin{document}

\maketitle

\section{Lecture - Part 1}

\subsection{Relationships Between Various Modes of Convergence}
\begin{enumerate}
    \item $X_n \xrightarrow{\text{D}} X \not\Rightarrow X_n \xrightarrow{\text{P}} X$
    
    \textbf{Example:}
    
    Let $X \sim unif(-1,1)$. This implies that $-X \sim unif(-1,1)$.
    
    \begin{equation*}
        X_n = \begin{cases}
            x & \text{ if $n$ is odd}\\
            -x & \text{ if $n$ is even}
        \end{cases}
    \end{equation*}
    
    Clearly, we have:
    \begin{equation*}
        X_n \xrightarrow{\text{D}} X
    \end{equation*}
    
    However, consider:
    \begin{equation*}
        |X_n-X| = \begin{cases}
            0 & \text{ if $n$ is odd}\\
            2|X| & \text{ if $n$ is even}
        \end{cases}
    \end{equation*}
    
    Let $\epsilon = 1$. Then:
    
    \begin{equation*}
        P(|X_n-X|<1) = \begin{cases}
            1 & \text{ if $n$ is odd}\\
            P(2|X|<1) = P(|X|<\frac{1}{2}) = \frac{1}{2} & \text{ if $n$ is even}
        \end{cases}
    \end{equation*}
    
    This shows the above: that convergence in \textit{distribution} \textbf{does not} imply convergence in \textit{probability.}
    
    \item \textbf{Theorem: }If $X_n \xrightarrow{\text{P}} X$, then $X_n \xrightarrow{\text{D}} X$. This essentially says that convergence in \textit{distribution} is the weakest kind of convergence we ahve talked about.
    
    \item \textbf{Theorem:} $X_n \xrightarrow{\text{P}} a \iff X_n \xrightarrow{\text{D}} a$
    
    \textbf{Proof:} 
    
    $(\Rightarrow)$: This follows from the above theorem. 
    
    $(\Leftarrow)$: Assume that $X_n \xrightarrow{\text{D}} a$. This means that:
    \begin{equation*}
        F_{X_n}(x) \rightarrow F_X(x) = \begin{cases}
            1 & x \geq a\\
            0 & x < a
        \end{cases}
    \end{equation*}
    
    Let $\epsilon>0$. We then look at the limit:
    \begin{equation*}
    \begin{split}
        \lim_{n\to\infty} P\left[|X_n - a| < \epsilon \right] &=\lim_{n\to\infty} P\left[a-\epsilon \leq X_n \leq a + \epsilon \right]\\
        &=\lim_{n\to\infty} F_{X_n}(a+\epsilon) - F_{X_n}(a-\epsilon)\\
        &= 1
    \end{split}
    \end{equation*}
    
    \item \textbf{Theorem:} Suppose that $X_n \xrightarrow{\text{D}} X$, and that $g$ is a continuous function on the support of $X$. Then $g\left(X_n\right) \xrightarrow[n \to \infty]{\text{D}} g(x)$. (The proof of this theorem requires measure theory)
    
    \item \textbf{Theorem: (Slutsky's Theorem)} Let $X_n \xrightarrow{\text{D}} X$ and $Y_n \xrightarrow{\text{P}} a$. Then:
    \begin{enumerate}
        \item $X_n Y_n \xrightarrow{\text{D}} aX$
        \item $X_n + Y_n \xrightarrow{\text{D}} X + a$
    \end{enumerate}
    
    \textbf{Example:} Show that:
    \begin{equation*}
        \frac{\sqrt{n}(\overline{X}_n - \mu)}{S_n} \xrightarrow{\text{D}} N(0,1)
    \end{equation*}
    
    If $\lim_{n\to\infty} Var(S_n^2) = 0$, then we have $S_n^2 \xrightarrow{\text{P}} \sigma^2$. This implies that:
    \begin{equation*}
        \Rightarrow \frac{S_n^2}{\sigma^2} \xrightarrow{\text{P}} 1 \Rightarrow \sqrt{\frac{S_n^2}{\sigma^2}} = \frac{S_n}{\sigma} \xrightarrow{\text{P}} 1
    \end{equation*}
    
    By the \textbf{Central Limit Theorem}:
    \begin{equation*}
        \frac{\sqrt{n}(\overline{X}_n -\mu) }{\sigma} \xrightarrow{\text{D}} N(0,1)
    \end{equation*}
    
    By \textbf{Slutsky's theorem}, if we multiply this by $\frac{\sigma}{S_n}$ (which converges to 1 as shown above), we get the Example result.
\end{enumerate}

\subsection{The Delta Method}

\textbf{Idea:} The general idea of the delta method is to approximate non-linear functions by their Taylor expansion (a  polynomial) and deal with them in an easier way. For example, if we had $\log(X)$ and wanted to write the expected value, , we don't have a good formula for it. But if we \textit{expand} it, and we know moments of $x$, we can approximate it.

Let $g(x)$ be a function with $n+1$ continuous derivatives. Then for a given real value $x_0$:
\begin{equation*}
\begin{split}
    g(x) &= g(x_0) + g'(x_0)(x-x_0) + \frac{g''(x_0)}{2!}(x-x_0)^2 + \dots \\
    &+ \frac{g^n(x_0)}{n!}(x-x_0)^n + \frac{g^{(n+1)}(\eta)}{(n+1)!}(x+x_0)^{n+1}
    \end{split}
\end{equation*}

Where $\eta$ is a number between $x$ and $x_0$. \textbf{Note:} This is \textit{not} an approximation to $g(x)$, but the answer itself (for some $\eta$).

\subsubsection{Linear Approximation}

Let's begin with a linear approximation of a function $g(x)$, around $x_0 = a$.

\begin{equation*}
    g(x) = g(a) + g'(a)(x-a) + o(|x-a|)
\end{equation*}

\noindent \textbf{Note:} In terms of \textit{convergence in probability}, we write:

\begin{equation*}
    Y_n = o_p(X_n) \iff \frac{Y_n}{X_n} \xrightarrow[n\to\infty]{P} 0
\end{equation*}

\subsubsection*{Examples:}
\begin{enumerate}
    \item Suppose we wanted to find $E[g(x)]$. Then, we would have:
        \begin{equation*}
            E[g(X)] \cong E[g(a) + g'(a)(x-a)]
        \end{equation*}
        
        \noindent In particular, let $a = E[X] = \mu$. Then:
        
        \begin{equation*}
        \begin{split}
            E[g(X)] &\cong E[g(\mu) + g'(\mu)(X-\mu)]\\
                &= g(\mu) = g(E[X])
        \end{split}
        \end{equation*}
        
        \noindent This says that if $g(x)$ is a function that is approximated well with a line near $x=\mu$, you can use this linear approximation. To find the variance, similarly we have:
        \begin{equation*}
            \begin{split}
                Var[g(X)] &= Var[g(\mu) + g'(\mu)(X-\mu)]\\
                    &= \left(g'(\mu) \right)^2 Var(X)
            \end{split}
        \end{equation*}
        
    \item Let $X_1,\dots,X_n \sim Bernoulli(p)$ (\textbf{iid}). Consider the \textit{sample proportion}, $\hat{p} = \frac{\sum X_i}{n}$.
    
    \begin{equation*}
        E[\hat{p}] = E[X_i] = p
    \end{equation*}
    
    To estimate the odds, $\frac{p}{1-p}$, we use $\frac{\hat{p}}{1-\hat{p}}$. The question is, what is our random variable? It is a \textit{non-linear} function of $\hat{p}$, so it is non-trivial to calculate the expectation. How would we go about computing the expected value? Consider:
    
    \begin{equation*}
        g(x) = \frac{x}{1-x}
    \end{equation*}
    
    First, let's find $E[g(x)]$. Consider $x = \hat{p}$. Then:
    
    \begin{equation*}
        E\left[\frac{\hat{p}}{1-\hat{p}} \right] \cong \frac{p}{1-p}
    \end{equation*}
    
    Now, to find the variance:
    
    \begin{equation*}
        \begin{split}
            g'(x) &= \frac{1}{(1-x)^2}\\
            \Rightarrow Var[g(x)] &\cong \left[g'(p) \right]^2 Var\left(\hat{p}\right)\\
            &= \left(\frac{1}{(1-p)^2} \right)^2\cdot \frac{p(1-p)}{n} = \frac{p}{n(1-p)^3}
        \end{split}
    \end{equation*}
    
    This estimator is \textit{approximately} unbiased, and, as $n$ gets large, the variance approaches $0$.
    
    \item Let $X \sim exp(1)$. Approximate $E[\log x]$ and $Var(\log x)$ using the \textit{delta method}. Well, we know that if we wanted to calculate the exact value of the expected value, we could find the ``true" value:
    \begin{equation*}
        E[\log x] = \int_0^\infty (\log x) e^{-x}dx \cong -0.5772
    \end{equation*}
    
    We know that
    \begin{equation*}
        E[X] = 1 \quad \text{ and } \quad Var(X) = 1
    \end{equation*}
    
    If we wanted to use the \textit{linear approximation}, we would have:
    \begin{equation*}
        \begin{split}
            E[g(X)] &\cong g\left(E[X] \right)\\
            &\cong \log 1 = 0
        \end{split}
    \end{equation*}
    
    So, let's use the \textbf{second order Taylor expansion}. We then have:
    \begin{equation*}
    \begin{split}
        g(x) &\cong g(\mu) + g'(\mu)(x-\mu) + \frac{g''(\mu)}{2}(x-\mu)^2\\
        g(x) &= \log x \quad g'(x) = \frac{1}{x} \quad g''(x) = -\frac{1}{x^2}\\
        \Rightarrow g(x) &\cong \log \mu + \frac{x-\mu}{\mu} - \frac{1}{2\mu^2}(x-\mu)^2\\
        E[g(x)] &\cong \log \mu - \frac{1}{2\mu^2} Var(x)\\
        E[\log x] &\cong \log 1 - \frac{1}{2} = -\frac{1}{2}
    \end{split}
    \end{equation*}
    
    Thus, we have shown that by increasing the order, we have vastly improved the approximation. If we want to find the \textbf{variance}, we will need to find the $4^{th}$ moment. We need to make an expansion here:
    \begin{equation*}
        g(x) = \log \mu + \frac{1}{\mu} (X-1) - \frac{1}{2\mu^2}(x^2-2\mu x + \mu^2)
    \end{equation*}
    
    Then, we have:
    \begin{equation*}
        \begin{split}
            Var[g(X)] &= \frac{1}{\mu^2} Var(X) - \frac{1}{4\mu^4} Var(X^2) + \frac{1}{\mu^2}Var(X)\\
            &- \frac{1}{2\mu^3}Cov(X,X^2) + \frac{1}{\mu^2}Cov(X,X) + Cov\left(-\frac{1}{2\mu^2}X^2,\frac{1}{\mu}X\right)
        \end{split}
    \end{equation*}
    
    With a bit of computation, the results follow.
    
\end{enumerate}
    
    \section{Lecture - Part 2}
    \subsection{The General Delta Method Theorem}
    \subsubsection*{Theorem:}
    
    Let $a_n$ be an increasing sequence of real numbers such that $a_n \to \infty$ and let $\theta$ be a fixed value. Furthermore, suppose that $g(x)$ is a continuously differentiable function such that $g'(\theta) \neq 0$. Then, if $Y_n = a_n(X_n-\theta) \xrightarrow[n\to\infty]{\text{D}} Y$ where $Y$ is a random variable, and $X_n$ is a sequence of random variables, then:
    
    \begin{equation*}
        a_n\left[g(X_n)-g(\theta) \right] \xrightarrow{\text{D}} g'(\theta)\cdot Y
    \end{equation*}
    
    \subsubsection*{Proof:}
    
    By Taylor's Theorem, we have:
    \begin{equation*}
        g(X_n) = g(\theta) + g'(w_n)(X_n-\theta)
    \end{equation*}
    
    Where $w_n$ is between $X_n$ and $\theta$
    
    \textbf{Facts:}
    \begin{enumerate}
        \item $X_n \xrightarrow{\text{P}} \theta$
        
        \textbf{Proof:} We know that $a_n(X_n-\theta)\xrightarrow{\text{D}} Y$, and that $\frac{1}{a_n}\xrightarrow{\text{P}} 0$. By \textbf{Slutsky's Theorem}, we have:
        \begin{equation*}
            \begin{split}
                \frac{1}{a_n}a_n(X_n-\theta) &\xrightarrow{\text{D}} 0\\
                \Rightarrow X_n - \theta \xrightarrow{\text{D}} 0 \Rightarrow X_n &\xrightarrow{\text{D}} \theta \Rightarrow X_n \xrightarrow{\text{P}}\theta
            \end{split}
        \end{equation*}
        
        \item $w_n \xrightarrow{\text{P}}\theta$ 
        
        \textbf{Proof:} Since $w_n$ is between $\theta$ and $x_n$, we know that
        
        \begin{equation*}
            |X_n-\theta| \geq |w_n -\theta|
        \end{equation*}
        
        This implies that
        \begin{equation*}
        \begin{split}
            P\left(|X_n - \theta| < \epsilon\right) &\leq P\left(|w_n - \theta| < \epsilon\right)\\
            \Rightarrow P(|w_n-\theta < \epsilon|) &\rightarrow 1
        \end{split}
        \end{equation*}
        
        \item Since $g'$ is \textit{continuous}, using \textbf{fact 2}, we have
        \begin{equation*}
            g'(w_n) \xrightarrow{\text{P}} g'(\theta)
        \end{equation*}
    \end{enumerate}
    
   \noindent Now consider, by Taylor's Theorem:
    \begin{equation*}
        a_n(g(X_n) - g(\theta)) = g'(w_n) a_n (X_n - \theta)
    \end{equation*}
    
    \noindent What do we know about this? Well $g'(w_n) \xrightarrow{\text{P}} g'(\theta)$, and $a_n(X_n-\theta) \xrightarrow{\text{D}} Y$. Thus we have, by \textbf{Slutsky's Theorem}:
    
    \begin{equation*}
        a_n(g(X_n) - g(\theta)) \xrightarrow{\text{D}} g'(\theta)Y
    \end{equation*}
    
    \subsubsection*{Theorem: (5.5.4)} Let $Y_n$ be a sequence of random variables that satisfy:
    \begin{equation*}
        \sqrt{n}(Y_n-\theta) \xrightarrow{\text{D}} N(0,\sigma^2)
    \end{equation*}
    
    \noindent for a given function $g$ and a specific value of $\theta$. Suppose that $g'(\theta)$ exists and $g'(\theta) \neq 0$. Then:
    
    \begin{equation*}
        \sqrt{n}[g(Y_n) - g(\theta)] \xrightarrow{\text{D}} N\left(0,\sigma^2\left(g'(\theta)^2 \right)\right)
    \end{equation*}
    
    For example, if $Y \sim N(0,\sigma^2)$, then we would have:
    \begin{equation*}
        g'(\theta)Y \sim N(0,\sigma^2 (g'(\theta))^2)
    \end{equation*}
    
    \subsubsection{Multidimensional Generalization}
    
    Let $g(\vec{x}): \mathbb{R}^n \rightarrow \mathbb{R}^1$. Then for any given vector $\vec{x_0} \in \mathbb{R}^n$:
    \begin{equation*}
        g(\vec{x}) = g(\vec{x_0}) + \nabla g(\vec{x_0})^T (\vec{x}-\vec{x_0}) + \frac{1}{2}(\vec{x}-\vec{x_0})^T H(\eta)(x-x_0)
    \end{equation*}
    
    Where $\eta$ is a value between $\vec{x}$ and $\vec{x_0}$, and where:
    \begin{equation*}
        \nabla g(\vec{x}) = \begin{pmatrix}
            \frac{\partial g}{\partial x_1} \\
            \vdots \\
            \frac{\partial g}{\partial x_n} 
        \end{pmatrix}
    \end{equation*}
    
    And:
    \begin{equation*}
        H(x)= \begin{pmatrix}
            \frac{\partial^2 g}{\partial x_1^2} & \cdots & \frac{\partial^2 g}{\partial x_1 \partial x_n} \\
            \vdots  & \ddots & \vdots\\
            \frac{\partial^2 g}{\partial x_n \partial x_1}  & \cdots & \frac{\partial^2 g}{\partial x_n \partial x_1}
        \end{pmatrix}
    \end{equation*}
    
   
    
   \noindent Let $\vec{X} = (X_1,...,X_n)^T$ be a random vector with mean $\vec{\mu} = (\mu_1,...,\mu_n)^T$. (\textbf{Note that:} $E[X_i] = \mu_i$). Then,
    
    \begin{equation*}
        g(\vec{X}) \cong g(\vec{\mu}) + \nabla g(\mu)^T(x-\mu)
    \end{equation*}
    
    \noindent We can now have:
    \begin{equation*}
        \begin{split}
            E[g(\vec{X})] &\cong g(\vec{\mu})\\
            Var[g(\vec{X})] &= \nabla g(\vec{\mu})^T Cov(\vec{X}) \nabla g(\vec{\mu}\\
          \text{Where, }  Cov(\vec{X}) &= \begin{pmatrix}
                \sigma_{11} & \sigma_{12} & \cdots & \sigma_{1n}\\
                \sigma_{21} & \sigma_{22} & \cdots & \sigma_{2n}\\
                \vdots & \ddots & \vdots & \vdots\\
                \sigma_{n1} & \sigma_{n2} & \cdots & \sigma_{nn}
            \end{pmatrix}
        \end{split}
    \end{equation*}
    
     \subsubsection*{Example:}
     
     The position of an aircraft relative to an observer is estimated by measuring the aircraft distance $R$ from the observer, and an angle $\Theta$ which is the angle from the observer to the aircraft, relative to the horizon.  We then have the \textit{altitude} of the aircraft $Y = R\sin (\Theta)$.
     
     Suppose that both $\Theta$ and $R$ are subject to random errors, and that $R$ and $\Theta$ are independent. Provided that $E[R] = 1000$, $E[\Theta] = \frac{\pi}{4}$, $Var(R) = 10$, and $Var(\Theta) = \frac{\pi^2}{100}$, obtain the approximate variance of the altitude of the plane. 
     
     First, we note that $g(R,\Theta):\mathbb{R}^2 \rightarrow \mathbb{R}^1$. Thus we have:
     
     \begin{equation*}
     \begin{split}
         g(R,\Theta) &= R \sin \Theta\\
         \nabla g(R,\Theta) &= \begin{pmatrix}
                \sin \Theta \\
                R \cos \Theta
            \end{pmatrix}\\
        \vec{\mu} &= \begin{pmatrix}
            E[R]\\
            E[\Theta]
        \end{pmatrix} = \begin{pmatrix}
            1000\\
            \frac{\pi}{4}
        \end{pmatrix}\\
        \Rightarrow \nabla g(\vec{\mu}) &= \begin{pmatrix}
            \sin \frac{\pi}{4}\\
            1000 \cos \frac{\pi}{4}
        \end{pmatrix}\\
        Cov(R,\Theta) &= \begin{pmatrix}
            Var(R) & Cov(R,\Theta)\\
            Cov(R,\Theta) & Var(\Theta)
        \end{pmatrix}\\
        Var(Y) &= \begin{pmatrix}
            \sin \frac{\pi}{4} & 1000 \cos \frac{\pi}{4}
        \end{pmatrix} \begin{pmatrix}  
                10 & 0\\
                0 & \frac{\pi^2}{100}
            \end{pmatrix} \begin{pmatrix}
                    \sin \frac{\pi}{4}\\
                    1000 \cos \frac{\pi}{4}
                \end{pmatrix}\\
                &= 10 \sin^2 \frac{\pi}{4} + 10^4 \pi^2 \cos^2 \frac{\pi}{4}
        \end{split}
    \end{equation*}
    

\end{document}